\documentclass[a4paper,11pt]{article}

% Standart.
\usepackage[T1]{fontenc}
\usepackage[utf8]{inputenc}
\usepackage{lmodern}
\usepackage[french]{babel}
\usepackage{textcomp}
\newcommand{\euro}{\texteuro}

% Page format.
\usepackage{geometry}
\geometry{top=2cm, bottom=2cm, left=1cm, right=1cm}

% Figures.
\usepackage{graphicx}
\graphicspath{{resources/}}
\usepackage{float}
\usepackage{array}
\usepackage{tikz}

% Math and chemistry.
\usepackage{chemist}
\usepackage{siunitx}
\usepackage{numprint}


% Links. Should be loaded last.
\usepackage{hyperref}
\hypersetup{colorlinks=true,
            linkcolor=blue,
            filecolor=magenta,
            urlcolor=cyan,
            citecolor=blue}

\title{Méthodologie ITIL}
\author{Notes par Benjamin Poncet et Hugo Verjus, Cours de Jean-Marc Chevereau}
\date{18 mars 2018}

\begin{document}

\maketitle
\tableofcontents

\

\section{Introduction}

\subsection{Enjeux}

Quel est le budget informatique d'une entreprise du CAC40?

Entre 1 et 3 milliards.

Combien d'applications chez France Télécom/Orange? Entre 20 000 et 25 000.

Entre 3 000 et 5 000 incidents par jour dans un grand système d'information et
résolution en 20 min en moyenne. Gagner 1 min représente 500 k\euro .

Il est donc important d'uniformiser les pratiques.

La moitié du temps est passé à faire des relances et à perdre du temps.

\subsection{Exemples}

\subsubsection{Hôtels}

Accor Hôtels

1967, Premier en France, 4 100 Hôtels, 95 pays, 240 000 employés.

Chiffre d'affaire: 6 milliards.

Capitalisation: 12 milliards.

\

Airbnb

2008, 3 millions de maisons, 192 pays, 2 000 employés., 65 000 villes.

Chiffre d'affaire: 2,8 milliards.

Capitalisation: 31 milliards.

\subsubsection{Voitures}

\begin{center}
  \begin{tabular}{|c|c|}
    \hline
    Tesla & Ford \\
    \hline
    2003 & 1903\\
    100 000 voitures/an & 6,8 millions de voitures/an\\
    Chiffre d'affaire: 7 milliards. & Chiffre d'affaire: 141 milliards.\\
    Capitalisation: 57 milliards. & Capitalisation: 49 milliards.\\
    \hline
\end{tabular}
\end{center}

\subsubsection{Banques}

Révolution avec la fin des agences, etc.

Changement de banques plus fréquents et faciles.

Révolut: Changements de monnaie moins chers, avec application smartphone. Zéro
frais quand on change. Carte mastercard liée à compte avec 3 monnaies (Euros,
livres, dollars). Créé par Polonais.

\subsubsection{General Electric}

\begin{quote}
``If you go to bed as an industrial company, you're going to make up as a
software company''
\end{quote}

\subsubsection{SMACS}

SMACS (Social Media Analytics Cloud Security) +15-20\% sur 5 ans.

\section{Devoteam}

Propagande.

\section{Définitions}

\subsection{Service}

Définition: A means of delivering value to customers by facilitating the
outcomes they want to achieve without the ownership of specific costs and risks.

Création de valeur (The basis of differentiation in the marketplace) =
Warranty (measured in terms of the levels of avalability, capacity, continuity
and security) + Utility ``What the customer gets'' (Measured on the basis of the
number of key ``outcomes supported'' and ``constraints solved'').

\

Costs and risks are still there, and managed by the service provider

\subsection{Processus}

A set of coordinated activities combining and implementing resources and
capabilities in order to produce an outcome, which, directly or indirectly,
creates value for an external customer or stakeholder.

\

\textbf{GROS TROU}

\

\subsection{ServiceNow}

Grosse entreprise d'ITIL.

\section{Request Fulfillment}

Il y a des requêtes standard, auxquelles j'ai toujours droit. Elles vont
dépendre de son entreprise et de son poste.

Les PCs sont renouvelés tous les 4-5 ans en général. Ils sont amortis sur 3 ans
(norme comptable). Au bout de ces 3 ans, la valeur comptable de ces ordinateurs
est de 0.

\subsection{TCO: Total Cost of Ownership}

Windows (ou pas), Antivirus, Écran, support, etc.

On calcule le coût total.

On a envie de prendre du matériel fiable, etc.

Rapport TCO / Prix d'achat: 3 (en général).

\subsection{Objectives}

Fournir un canal pour gérer les requêtes standard pour lesquelles une réponse
pré-définie existe.

Fournir de l'information aux utilisateurs et aux clients sur la disponibilité
des services et de la procédure pour les obtenir.

Organiser la réponse aux requêtes. Ex: Organiser le travail pour l'arrivée d'une
nouvelle personne dans l'entreprise.

Avoir les informations générales sur la gestion de l'information, plaintes et
demandes.

\subsection{Concepts}

Service Request\\
A request from a user for information or advice, or for a standard change
(password reset, provide stadard IT service to a new user...)

Request Model \\
A model of request which typically includes some form of pre-approval by change
management

Not the overship of SR(s resides with the service desk

\end{document}