\documentclass[a4paper,11pt]{article}

% Standart.
\usepackage[T1]{fontenc}
\usepackage[utf8]{inputenc}
\usepackage{lmodern}
\usepackage[french]{babel}

% Defining sane commands for euros.
\usepackage{textcomp}
\newcommand{\euro}{\texteuro}

% Page format.
\usepackage{geometry}
\geometry{top=2cm, bottom=2cm, left=1cm, right=1cm}

% Figures.
\usepackage{graphicx}
\graphicspath{{resources/}}
\usepackage{float}
\usepackage{array}
\usepackage{tikz}

% Math and chemistry.
\usepackage{chemist}
\usepackage{siunitx}
\usepackage{numprint}


% Links. Should be loaded last.
\usepackage{hyperref}
\hypersetup{colorlinks=true,
            linkcolor=blue,
            filecolor=magenta,
            urlcolor=cyan,
            citecolor=blue}

\title{partie de GREF}
\author{Bashar DUDIN}
\date{}

\begin{document}

\maketitle

\subsection*{Disclaimer}

Ce sont des notes de cours imparfaites qui n'ont absolument rien d'officiel.
Le professeur n'a eu strictement aucune implication dans la création de ce
fichier. Il n'engage personne et en particulier ni le professeur ni les
étudiants qui l'ont écrit.

C'est juste des élèves quelconques qui font de leur mieux pour suivre en cours
et vous partager leurs notes pour vous aider dans vos révisions.

Sur ce, bon travail :).

\tableofcontents
\newpage
\section{Infos}
You can get many infos there.
\url{https://github.com/bashardudin/GraphsAndFlows}

\section{Recap}
\paragraph{shortest path}
\begin{itemize}
  \item single-pair (svg dst) shortest path problem
  \item single src any destination : shortest path problem
  \item Any pair shortcut path pb
\end{itemize}

\paragraph{algos}
\begin{itemize}
  \item Bellman-Ford [general no negative valued edges]
  \item Dijkstra [positive weights]
  \item Bellman  [no cycles]
  \item Floyd warshall 
\end{itemize}



\subsection{Safest path}
% insert image here
Aim : safest path along which to go.

\

To use shortest paths algorithms for it  is about apllying shortest paths to graphs having  -log(above weights).

\subsection{Finding paths}
To find critical paths going back from ending point you look for predecessors whose earliest starting date  is earliest starting point 'E' - time task takes. And you go through recursively.

\paragraph{margins} \hfill \break
It is the difference between earliest starting date of task and the latest (not delaying project).

\paragraph{free margin} \hfill \break
Is the difference between earliest starting date \& latest one not modifying the earliest starting date of any successors.\\

\subsection{A bit of formalism}

Let $ c : R_+$ be the capacity in a given graph $G$  (having s anf t)\\
Then a flow on $G$ is a function $f:A \rightarrow R_+$\\
Satisfying :
(1) $\forall a \in A m f(a) <=  c(a) $\\
(2) $\forall i \in V|\{s,t\}$ :\\
$\sum f(a) = \sum f(a)$\\
$a \rightarrow v \quad \quad v \rightarrow a$\\
$ a \in A \quad \quad a \in A$\\
\\

\subsection{Exercice 6 (exercise sheet)}
1)\\
\includegraphics[width = 18cm]{graph_first.png}
2)
Simply apply "improving path" algo and you're done.


\end{document}
