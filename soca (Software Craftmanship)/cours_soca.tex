\documentclass[a4paper,11pt]{article}

% Standart.
\usepackage[T1]{fontenc}
\usepackage[utf8]{inputenc}
\usepackage{lmodern}
\usepackage[french]{babel}
\usepackage{eurosym}

% Page format.
\usepackage{geometry}
\geometry{top=2cm, bottom=2cm, left=1cm, right=1cm}

% Figures.
\usepackage{graphicx}
\graphicspath{{resources/}}
\usepackage{float}
\usepackage{array}
\usepackage{tikz}

% Math and chemistry.
\usepackage{chemist}
\usepackage{siunitx}
\usepackage{numprint}
\usepackage{amsmath}
\usepackage{amssymb}
\usepackage{stmaryrd}

% Links. Should be loaded last.
\usepackage{hyperref}
\hypersetup{
  colorlinks=true,
  linkcolor=blue,
  filecolor=magenta,
  urlcolor=cyan,
  citecolor=blue}


% Dummy environment. Useful for some convoluted layouts.
\newenvironment*{dummyenv}{}{}

% Custom commands
\newcommand{\R}{\mathbb{R}}
\newcommand{\Z}{\mathbb{Z}}
\newcommand{\N}{\mathbb{N}}
\newcommand{\der}{\,\mathrm{d}}
\newcommand{\e}{\mathrm{e}}
\newcommand{\ti}{\cdot}

\newcommand{\ls}{\begin{itemize}}
\newcommand{\li}{\item}
\newcommand{\lend}{\end{itemize}}

\title{Software Craftmanship}
\author{Emmanuel Chaffraix}
\date{25 avril 2018}

\begin{document}

\maketitle

\subsection*{Disclaimer}

Ce sont des notes de cours imparfaites qui n'ont absolument rien d'officiel.
Le professeur n'a eu strictement aucune implication dans la création de ce
fichier. Il n'engage personne et en particulier ni le professeur ni les
étudiants qui l'ont écrit.

C'est juste des élèves quelconques qui font de leur mieux pour suivre en cours
et vous partager leurs notes pour vous aider dans vos révisions.

Sur ce, bon travail :).

\tableofcontents

\newpage

\section{Soutenance}

Le 3 juillet.

Soutenance: 10min sur un sujet donné par le prof

10min de restitution de cours.

Par groupe de 3.

Slides autorisées mais non nécessaires.

\section{Introduction}

\subsection{Prof}

Ex-GISTR 2009.

\subsection{Iceberg}

La tech n'est que la partie émergée de l'iceberg.

SLA: qualité de service. (uptime, etc)

\section{Clean code}

Interesting read.

\subsection{Qu'es-ce que du clean code?}

\begin{itemize}
\item Bugs
\item Coding style
\item Travailler avec les autres (50, 100, 1000 personnes)
\end{itemize}

\subsection{La vie du développeur}

Jamais remercié, toujours des plaintes!

\

Par code:

\begin{itemize}

\item Fail compilation

\item Segfault

\item Stack overflow

\end{itemize}

\

Client:

\begin{itemize}

\item ASAP

\item Bad UX

\item Not working.

\end{itemize}

\subsection{... est géniale}

On créé des produits

On décrouvre des activités

On apprend!

\subsection{Software Craftmanship Manifesto}

Pas seulement du logiciel qui marche, mais aussi du logiciel bien conçu.

Pas seulement répondre à du changement mais ajouter constemment de la valeur

Pas seulement des individus et des interactions mais une communauté de
professionnels

Pas seulement une collaboration avec des clients, mais aussi des partenariats
productifs.

\subsection{Exemple}

Éviter les nombres magiques, la duplication de code, les abbréviations
non-évidentes. Prendre des noms clairs.

Quelle unité? \euro? £? \$?

C'est de la dette technique.

\subsection{L'impact de la dette technique}

Lisibilité du code.

Développement de nouvelles fonctionnalités.

Correction de bugs.

\

C'est finalement le calcul de son impact en tant que développeur sur un projet.

\subsection{Quelques principes généraux}
\subsubsection{Be KISS}

Keep It Simple, Stupid.

Exemple: Le bon coin. Très peu de fonctionnalités.

\subsubsection{Be DRY}

Don't repeat yourself.

Pas de code similaire.

\subsubsection{Be focused}

YAGNI: You Ain't Gonna Need It.

Ne pas développer des trucs qu'on ne va pas utiliser.

\subsubsection{DIE}

Duplication Is Evil.

Copy/Paste is easy, but hard to bug fix.

\subsection{Quelques principes OO}

Encapsulation

Héritage

Polymorphisme

\begin{itemize}
\item Overloading
\item Templates
\item Subtypings
\end{itemize}

Liste: souvent tableaux mais avec des éléments de liste chaînée.

\subsubsection{SOLID}

\begin{itemize}
\item SRP: Single responsibility principle

  Une classe ne doit avoir qu'une seule raison de changer.

  Ne vérifier un email qu'à un seul endroit, par exemple.

  Séparer les actions effectuées par une fonction.

  \
\item OCP: Open/closed principle

  Les entités logicielles doivent être ouvertes à l'extension mais fermées à la
  modification.

  Exemple: gros switch case à modifier.

  Solution: stocker le résultat dans chacune des classes.

  \

\item LSP: Liskov substitution

  Si S est un sous-type de T, alors les objets de type T peuvent être remplacés
  par des objets de type S sans changer aucune des propriétés désirables du
  programme.

  Si on manipule les types, on n'utilise pas bien son langage OO.

  \

\item ISP: Interface Segregation Principle

  Many client-specific interfaces are better than one general-purpose interface.

  \

\item DIP: Dependency Injection Principle

  One should ``depend upon abstractions, not concretions''.

  On ne doit dépendre que des abstractions de classes, pas des classes
  elles-mêmes.

  Exemple: Si on a un loggeur partout, faire une interface qui permet de tout
  changer facilement.

\end{itemize}

\section{Soyez Agiles}

\subsection{V Cycle}

Compléter.

Niveau fonctionnel: Concept development \quad Transition Operation \& maintenance.

Niveau system: Requirements Engineering. \quad Test \& evaluation

Niveau Subssytem

Bien pour les projets longs où on a besoin de s'engager sur un résultat.

\subsection{Agile}

Meilleure satisfaction client: il a quelque chose d'utilisable tout le long.

Voir \url{https://www.youtube.com/watch?v=3wyd6J3yjcs}.

Voir les diapos.

Idea $\underset{Build}{\longrightarrow}$ Product $\underset{Measure}
{\longrightarrow}$ Data $\underset{Learn}{\longrightarrow}$ Idea.

Voir l'entreprise qui répondait à des questions ouvertes en 2007, rachetée par
Google. Ils ont complètement faké le truc: Mail, gestion de ticket, moteur de
recherche sur les questions et mail sinon. Ils n'ont pas eu le temps de finir
leur algo.

Financement de Critéo avec de la vente de céréales.

Méthode: Existe depuis les années 60 avec Ford et Toyota.

Méthode lean.

\subsection{SCRUM}

\begin{itemize}

\item Product owner

  Doit pouvoir lister les fonctionnalités qu'il souhaite et comprendre les
  besoins du client.

  \

\item Scrum master

  Est responsable de l'équipe (souvent un dev). Change régulièrement. Pas
  forcément le chef.

  \

\item Developer team

  \

\item Backlog

  Do to list sur le projet. Ne fait que grossir. C'est tout ce que souhaite le
  client.

  \

\end{itemize}

Poker planning: on met un prix / complexité sur chacune des fonctionnalités et
on discute de quand on les implémente.

\subsubsection{SCRUM Workflow}

\begin{itemize}
\item Sprint meeting planning

  Remplir le backlog. On chosit ce qu'on va implémenter.

  \

\item Daily Scrum meeting

  Une fois par jour, état de l'avancement du projet, debout.
  On dit ce qu'on a fait, ce qu'on va faire et là où on bloque.

  \

\item Sprint retrospective

  Voilà ce qu'on a fait par rapport à ce qu'on avait prévu.
\end{itemize}

\subsubsection{Scrum diagram}

Voir le diagramme sur les déiapos du prof.

\subsection{Kanban}

Voir le diagramme.

Chacun a une couleur de ticket.

Intérêt: on limite le nombre de tickets par personne et par colonne.

\section{Think test}

\subsection{Intro}

TDD: Test driven approach

BDD: Behaviour driven approach

\subsection{Why do we test?}

Tests help to understand requirements

Tests pretoects future developements

Tests are a fall protection

We are human after all.

\subsection{How do we test?}

\textbf{We create code to test our code}

\

\begin{itemize}

\item Unit test: Focus on ONE function usage, mock dependencies. Ex: base de
  données: on en prend une fausse.

  Developer.

\item Integration test: crosses the boudnary between components.

  Developer.

\item Behaviour test: An example of the user using the system. Test a usecase

  Scrum master + product owner.

\end{itemize}

\subsection{TDD: Test Driven Development}

No code without a test.

\begin{enumerate}
\item Write a test that fails
\item Write the code that fulfills the test.
\item Refactor source code (feature and test !)
\end{enumerate}

Compliqué à mettre en place pour les personnes qui ne sont pas habituées à faire
du test.

Ralentit le développement.

\subsection{BDD: Behaviour Driven Development}

A simple test: Given... When... Then...

\textbf{Given} a user with an account

\textbf{When} he tries to create a new account with existing account credentials

\textbf{Then} it must be logged in with the existing account

A feature contains tests.

Voir \url{https://cucumber.io/}

\subsection{Think test}

Write a failing feature test

\subsection{The double loop}

\begin{figure}[h]
  \centering
  \includegraphics[width = 0.6 \textwidth]{bdd-and-tdd-cycle.png}
  \caption{The double loop}
\end{figure}

\subsection{Test pyramid}

UI > Service > Unit.

In time and money.

\subsection{What is a good test?}

Easy to understand, clear when fails.

Determinist

Focused

AAA: Arrange, Act, Assert.

\subsection{Test coverage}

A 100\% coverage is not a bug-free code.

\section{From dev to prod}

\subsection{Working with others}

Avoid:

\ls
\li Missing code files
\li Ugly code naming
\li Code deletion
\lend

Anticipate problems

So rules are required!
\ls
\li Coding style
\li Code reviewal
\li Automated process
\lend

Infinite loop: Code, Build, Test, Release, Deploy, Operate, Measure, Plan.

\section{Software architectures}

\subsection{Monolyth}

Fine for simple projects.

\subsection{Multi-tier}

\subsubsection{3-tier}

Presentation

Logic (for the job ex: age calculation).

Data.

\

Typically for local software.

\subsection{SOA: Service Oriented Architecture}

\subsection{CRUD}

\subsection{REST}

Standard API: swagger.

\subsection{The current state problem}

\subsection{Event sourcing}

Cucurrent environment (multithreading, processing, etc): try to work with
immutable objects (perf, corruption).

\subsection{CI/CD Quesaco}

Continuous: forming an unbroken whole; without interruption

Integration: the coordiantion of processes

Delivery: The actions of delivering lettres, packages, ... babies

Deployment: The action of bringing resources into usefulness.

\subsubsection{CI process goal}

Ex: compile when someone pushes.

\subsection{How to go live}

\subsection{Delivery process}

\section{Environments}

\subsection{Dev}

\subsection{Prod}

\subsection{Pre-prod}

\subsection{UAT}

User acceptance test.

\subsection{Where to push?}

Create packages for each component.

\subsection{CI/CD Tools}

Shell script!

Jenkins, Tavin CI, Teamcity...

\section{Design patterns}

\subsection{Definition}

\subsection{GoF: 24 patterns}

Book: design pattern.

\subsubsection{Creational}

\paragraph{Singleton}

Restricts the instanctionation of a class to one object.

Saves space.

Have a single way to interact with an application.

The singleton is instanciated only when needed.

Thread-safe: double-checking.

\paragraph{Factory Method}

Creates the actual classes for you.

\paragraph{Adapter (aka. Wrapper)}

Making an interface for every langauge call you make.

\paragraph{Decorator}

Modify the behavior of a Banana without changing Banana itself.

\subsubsection{Behavioural}

\paragraph{Visitor}

You know it.

\paragraph{Observer}

Do a notify.

\paragraph{Strategy}


\end{document}
