\documentclass[a4paper,11pt]{article}
\usepackage[T1]{fontenc}
\usepackage[utf8]{inputenc}
\usepackage{lmodern}
\usepackage[french]{babel}

% Defining sane commands for euros.
\usepackage{textcomp}
\newcommand{\euro}{\texteuro}

% Page format.
\usepackage{geometry}
\geometry{top=2cm, bottom=2cm, left=1cm, right=1cm}

% Figures.
\usepackage{graphicx}
\graphicspath{{resources/}}
\usepackage{float}
\usepackage{array}
\usepackage{tikz}

\usepackage{amsmath,amsfonts,amssymb}


% Links. Should be loaded last.
\usepackage{hyperref}
\hypersetup{colorlinks=true,
            linkcolor=blue,
            filecolor=magenta,
            urlcolor=cyan,
            citecolor=blue}



\title{Transformee Fourier}
\author{}

\begin{document}

\maketitle
\tableofcontents
\newpage

\section{Historique}

Dans la musique, on peut retrouver des gammes 'non-dissonantes'.\\
Les instruments etaient accordes a l'oreille avec une note de base.\\
Cela creait des dissonnances.\\

\section{Introduction}
Peut on obtenir un signal elementaire avec des signaux elementaire ?\\

On va s'interesser aux conditions qui font qu'un signal est compose de signaux elementaires.

\begin{align*}
f(t)  \\
        f   ('periode')  2\pi \\
\end{align*}

\subsection{Definition}

\begin{align*}
  e^{\alpha z}   (derivee) &\rightarrow \alpha e^{\alpha z}\\
  e^{\alpha z} (primitive) &\rightarrow \frac{1}{\alpha} * e^{\alpha z}\\
\end{align*}

\begin{align*}
  g(t) &= \sum_{k \in Z} e_{n}e^{int}\\
  \int_{0}^{2\pi} g(t)e^{-int} dt &= \int_{0}^{2x} \sum_{m \in Z} e_m e^{imt} e^{imt} dt \\
  &= \sum_{m \in Z} c_m \int_{0}^{2\pi} e^{i(m - n)t} dt\\
  &= c_n \int_{0}^{2\pi} + \sum_{m \neq n} e^{i(m-n)t} dt \\
  &= 2\pi c_n + 0
\end{align*}

\begin{align*}
  c_n = \frac{1}{2\pi} \int_0^{2\pi} g(t) e^{-int} dt
\end{align*}
\subsection{Definition 2}
Une fonction periodique est une somme de fonction elementaires.

\newpage
\subsection{Definition 3}
Prenons $f$ de periode $2\pi$\\
La serie de Fourier de $f$ est :\\

\begin{align*}
b_0 + \sum_{n = 1 }^{\inf} b_n  \cos(nt) + \sum_{n = 1}^{\inf} a_n \sin(nt)\\
\sum_{n \in Z} c_n e^{int}\\
\end{align*}

\begin{align*}
  b_0 &= \frac{1}{2\pi} \int_{0}^{2\pi} f(t) dt\\
  b_n &= \frac{1}{\pi} \int_{0}^{2\pi} f(t) \cos(nt)dt\\
  a_n &= \frac{1}{\pi} \int_0^{2\pi} f(t)\sin(nt) dt\\
  c_n &= \frac{1}{2\pi} f(t) e^{-int} dt
\end{align*}
$n > 0$ \\
\begin{align*}
  c_n &= \frac{b_n + i a_n}{2}\\
   c_{-n} &= \frac{b_n - i a_n}{2}\\
  \end{align*}
   $n = 0$\\
  \begin{align*}
 c_0 &= b_0
\end{align*}

$f$ a valeur nulle.\\
$a_n$ et $b_n$ reels\\

$f$ est paire\\
$ \forall a_n = 0$\\
$f$ est impaire\\
$\forall b_n = 0$\\

Ensemble de fonctions periodiques :\\
- continue\\
- continent derivable\\
- Integrale\\
- "Pas trop de  discontinuite"\\
- ...\\

Un ensemble $E$ choisi.\\
Espace vectoriel.\\
$f$ et $g \in $ E\\
$\alpha_1 \beta \in C$\\
$\alpha f + \beta g \in E$\\

\begin{align*}
  <f,g> &= \frac{1}{2\pi} \int f(t) \overline{g}(t) dt\\
\end{align*}

\begin{align*}
<f,g> &= \frac{1}{2\pi} \int f\overline{f}(t) dt\\
      &= \frac{1}{2\pi} \int |f|^2 dt
\end{align*}
ressemble a $||f||^2$

\begin{align*}
  <e^{int},e^{int} > &= 0\\
\end{align*}
 $n \neq m$\\
 \begin{align*}
   <e^{int},e^{imt}> &= 1\\
 \end{align*}
 
 \begin{align*}
\sum_{n \in Z} < f,e^{in}> e^{int}\\   
 \end{align*}

En dimension 3,$ e_1, e_2, e_3 $etant des espaces vectoriels, on peut exprimer $u(x,y,z)$, tq:\\
$u = (u.e_1)e_1 + (u.e_2)e_2 + (u.e_3)e_3$\\

\section{Convergence}

\subsection{Theoreme}
f  classe $C^2$\\
Convergence\\
$f(t) = \sum c_n e^{int}$\\

Fonction "reguliere par morceau"\\
- Un nombre fini de discontinuites\\
- nombre fini d'extrema\\

Si $f$ continu en $t$ ; $f(t) = \sum c_{n} e^{int}$\\
Si $f$ discontinu en $t$ : $\sum c_n e^{int} = (\frac{1}{2} (f(t-) + f(t+)))$\\

\subsection{Phenomene de Gibbs.}

\subsubsection{Integration et derivation}
\begin{align*}
 \sum_a^b e^{int} dt &= (b - a) c_0 + \sum_{n \neq 0} C_n \int_a^b e^{int} dt\\
 &= (b - a) c_0 + \sum_{n \neq 0 } \frac{C_n}{in} (e^{inb} - e^{ina}) 
\end{align*}

Pour la derivation:\\
Hypothese $f$ continu derivable\\
$f'$ regulier par morceaux\\
\begin{align*}
  \sum inc_n e^{int} &= f'(t) ... continue\\
&=\frac{1}{2} (f'(t+)+ f'(t-)) ... discontinue
\end{align*}

\subsubsection{Exemple}

\begin{align*}
  f(t) = 1 si t \in ]-a, +a[\\
        0 ailleurs sur [-\pi,\pi]\\
        ...\\
        f(t) = \frac{a}{\pi} + \frac{2}{\pi} \sum_{n = 1}^{\inf} \frac{\sin(n a)}{n} cos(nt)\\
\end{align*}

\subsubsection{Cas particulier}
\begin{align*}
f(t) &= \frac{1}{2} + \frac{2}{\pi}\sum_{R = 0}^{\inf} \frac{(-1)^k}{2k + 1} \cos{(2k +1) }t  \\
f(t) &= \frac{1}{2} + \frac{2}{\pi} \sum_{k = 0}^{\inf} \frac{1}{2k =1} \sin((2k +1)t)\\
\end{align*}

\section{Formulaire}
\subsection{Identite de Parseval}

\url{http://www.bibmath.net/dico/index.php?action=affiche&quoi=./p/parseval.html}

\subsection{Densite Gausienne}

\begin{align*}
  f(x) = \frac{1}{\sigma \sqrt{2\pi}} e^{-\frac{1}{2}\frac{x^2}{\sigma^2}}\\
  F(f)(u) = e^{- \frac{1}{2} \sigma^2 u^2}
\end{align*}


\subsection{Translation}
\begin{align*}
  F(f(x-a))(u) = e^{iau} F(f)(u)
\end{align*}

\subsection{Modulation}

\begin{align*}
F(e^{i\omega_0 x} f(u))    = F (u - \omega_0)
\end{align*}

\subsection{Changement d'echelle}
\begin{align*}
  F(f(ux)) = \frac{1}{|a|} F(\frac{u}{a})
\end{align*}

\subsection{Conjugaison}
\begin{align*}
  F(\overline{f}(x)) = \overline{F(f)}(-u)
\end{align*}

\subsection{Derivation}
\begin{align*}
  F(f)(u) = \int f(x)e^{-iux} dx\\
  F(f')(u) = i u F(f)(u)
\end{align*}

\subsection{Theoreme de Parseval}

\begin{align*}
  \int_{-\infty}^{\infty} |f(x)|^2 dx = \frac{1}{2\pi} \int_{-\infty}^{\infty} |F(u)|^2 du\\
  \int_{-\infty}^{\infty} F(u)\overline{G}(u) du\\
\end{align*}

\subsection{Convolution}

\begin{align*}
  f * g(x) = \int_{x \in R} f(u) g(x - a) du\\
  F(f * g) (u) = F(f)(u) . F(g)(u)\\
\end{align*}
De plus si $f,g \in L^2$
\begin{align*}
  F(fg) = F(f) * F(g)
\end{align*}






\end{document}
